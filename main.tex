\documentclass{article}
\usepackage{graphicx} % Required for inserting images

\title{Game Motion Control}
\author{Damian Rogalski}
\date{May 2024}

\begin{document}

\maketitle

\section{Wprowadzenie}
Tematem mojego projektu jest sterowanie w grach za pomocą ruchu dłoni. Projekt podzielony jest na 3 częsci. Każda z nich posiada swoją odrębną grę demonstracyjną, żeby zwizualizować moje zaimplementowane rozwiązanie konkretnego problemu.

\section{Gry}
\subsection{Maze}
Gra Maze to prosta gra logiczna, gdzie gracz za pomocą ruchu dłoni (lewo, prawo, góra, dół) wskazuje kierunek, w którym ma się poruszyć zielony kwadrat. Celem gry jest dotarcie do czerwonego kwadratu, który jest metą. Gracz do dyspozycji ma trzy mapy. Podczas rozgrywki gracz może zbierać opcjonalne niebieskie punkty. Gra obrazuje działanie detekcji położenia dłoni w przestrzeni z wykorzystaniem biblioteki MediaPipe.

\subsection{Cursor Game}
Gra Cursor Game to prosta gra, w której gracz może za pomocą palca wskazującego kontrolować kursor na ekranie i ma za zadanie najeżdżać nim na czerwone kwadraty losowo generowane na ekranie. Za każdy wskazany obiekt gracz otrzymuje punkt. Na wskazanie obiektu gracz ma ograniczony czas. Co dziesięć zebranych punktów kwadraty zmniejszają się, żeby utrudnić graczowi rozgrywkę. Gra ilustruje działanie detekcji palca wskazującego i obliczania jego pozycji w przestrzeni. Dodatkowo program oblicza odległość pomiędzy palcem wskazującym a kciukiem, gdyż gest ich złączenia ze sobą służy do zresetowania rozgrywki.

\newpage

\subsection{Numbers Game}
Gra Numbers Game wyświetla użytkownikowi na ekranie liczbę od 1 do 4, a gracz musi wskazać losowo wygenerowaną liczbę palcami. Przy poprawnym wskazaniu gracz otrzymuje informację "Correct!" a po dwóch sekundach generowana jest kolejna liczba. Rozgrywka ma za zadanie pokazanie działania predykcji mojego wytrenowanego modelu.


\section{Model Klasyfikujący}
\subsection{Dane}
Projekt posiada własny model klasyfikujący. Jako dane treningowe i testowe wykonałem 400 zdjęć dłoni wskazującej liczbę od 1 do 4 (po 100 na klasę). Do przygotowania zdjęć stworzyłem program, który samemu wykrywa dłoń korzystając z biblioteki cvzone, następnie odpowiednio przycina zdjęcie do wskazanego wymiaru i je zapisuje.

\subsection{Trenowanie}
Model wytrenowałem na danych w proporcjach 80\%/20\% (train/test). Dla 10 epok model trenuje się bardzo szybko, a jego dokładność wynosi około 95\%.

\section{Testowanie}
\subsection{Maze}
Przy testowaniu gry szybko wyszły ograniczenia związane z precyzją ruchu. Dla zmaksymalizowania komfortu rozgrywki należy zachować odpowiedni dystans (najlepiej około 1 metr) od kamery, żeby ułatwić detekcję pozycji dłoni. Przy zachowaniu odpowiednich warunków oświetleniowych gra nie generuje większych problemów.

\subsection{Cursor Game}
W tej grze najmocniej wyłoniły się ograniczenia związane z biblioteką MediaPipe oraz sprzętem. Kursor na ekranie nie wykonuje płynych przejść a skokowe. Stanowi to pewien dyskomfort, jednakże rozgrywka nadal generuje sporo frajdy i generuje sporo frajdy przy zachowaniu dobrego oświetlenia.

\newpage

\subsection{Numbers Game}
Ta gra przyciągneła największą uwagę, gdyż korzysta ona z mojego własnego modelu klasyfikującego. Przy testach okazało się, że program bardzo często błędnie klasyfikuję jedynkę jako dwójkę. Ponadto z uwagi na zbyt niską liczbę danych program nie radzi sobie z detekcją gestów lewą ręką oraz ma mieszaną detekcję gestów u innych osób. Oświetlenie również odgrywa poważną rolę przy predykcji gestu. Jednakże, gdy ja testuję program to w wiekszości przypadków działa on poprawnie.

\section{Podsumowanie}
Program w interesujący sposób demontruje działanie różnych narzedzi do detekcji dłoni. Efekty momentami są mizerne, żeby chwilę później stać się satysfakcjonującymi. Dużo problemów wynika z ograniczeń sprzętowych, czasowych czy samych narzedzi. Niemniej osoby testujące program były zadowolone i mile spędziły czas podczas rozwoju programu.


\end{document}
